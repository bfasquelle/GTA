\documentclass{beamer}
  \usepackage[utf8]{inputenc}
  \usepackage{graphicx}
  \usepackage{array}
  \usepackage{url}
  \usetheme{Warsaw}

  \title{Bertrand competition in networks}
  \author{Benjamin Fasquelle, Arthur Queffelec, Raphaël Truffet}
  \institute{École Normale Supérieure de Rennes, département Informatique et Télécommunications}

  \begin{document}

\setbeamercolor{structure}{fg=magenta}

\AtBeginSection[]
{
  \begin{frame}<beamer>
    \frametitle{Sommaire}
{\scriptsize\tableofcontents[currentsection]
}
  \end{frame}
}

\addtobeamertemplate{footline}{\hfill\insertframenumber/\inserttotalframenumber}

  \begin{frame}
  \titlepage
  \end{frame}


\begin{frame}
  \tableofcontents
  \end{frame}


\section{Introduction and motivations}

\begin{frame}
there have been growing concerns about the effective-
ness of its current routing protocols in finding good routes and ensuring quality
of service. Congestion and QoS based pricing has been suggested as a way of
combating the ills of this distributed growth and selfish use of resources (see,
e.g., [5, 7, 8, 10, 12]). Unfortunately, the effectiveness of such approaches relies on
the cooperation of the multiple entities implementing them, namely the owners
of resources on the Internet, or the ISPs. The ISPs’ goals do not necessarily
align with the social objectives of efficiency and quality of service; their primary
objective is to maximize their own market share and profit.
\end{frame}



\begin{frame}
given a large combinatorial
market such as the Internet, suppose that the owners of resources selfishly price
their product so as to maximize their profit, and consumers selfishly purchase
bundles of products to maximize their utility, how does this effect the functioning
of the market as a whole?
\end{frame}


\begin{frame}
Economists have traditionally studied the properties of equilibria that emerge
in pricing games with competing firms in single-item markets (see, e.g., [15, 16]
and references therein). It is well known [11], e.g., that in a single-good free
market, oligopolies (two or a few competing firms) lead to a socially-optimal
equilibrium 3 . On the other hand, a monopoly can cause an inefficient allocation
by selfishly maximizing its own profit. Fortunately the extent of this inefficiency
is bounded by a logarithmic factor in the (multiplicative) disparity between
consumer values, as well as by a logarithmic factor in the number of consumers.

These classical economic models ignore the combinatorial aspects of network
pricing, namely that consumers have different geographic sources and destina-
tions for their traffic, and goods (i.e., edges) are not pure substitutes, but rather
are a complex mix of substitutes and complements, as defined by the network
topology.
\end{frame}


\begin{frame}
which properties of stan-
dard price equilbrium models carry over to network/combinatorial settings? For
example, are equilibria still guaranteed to exist? Are equilibria fully efficient?
Does the answer depend in an interesting way on the network/demand struc-
ture? The network model captures the classical single-item setting in the form of
a single-source single-sink network with a single edge (modeling a monopoly), or
multiple parallel edges (modeling an oligopoly). In addition, we investigate these
questions in general single-source single-sink networks, as well as multiple-source
single-sink networks.
\end{frame}



\section{Network Pricing Game and Bertrand Competition}

\subsection{Network Pricing Game model}

\begin{frame}
A network pricing game (NPG) is characterized by a directed graph G = (V, E)
with edge capacities {c e } e E , and a set of users (traffic matrix) endowed with
values. Each edge is owned by a distinct ISP. (Many of our results can be easily
extended to the case where a single ISP owns multiple edges.) The value as-
sociated with each chunk of traffic represents the per-unit monetary value that
the owner of that chunk obtains upon sending this traffic from its source to its
destination. User values are represented in the form of demand curves 4 , D (s,t) ,
for every source-destination pair (s, t), where for every l, D (s,t) (l) represents the
amount of traffic with value at least l. When the network has a single source-sink
pair, we drop the subscript (s, t). We use D to denote the “demand suite”, or
the collection of these demand curves, one for each source-sink pair. Without
loss of generality, the minimum value is 1, that is, D (s,t) (1) = F tot
s,t for all pairs
(s, t), and we use L to denote the maximum value—L = sup{l|D (s,t) (l) > 0}.
\end{frame}



\subsection{Prices of anarchy and stability}

\begin{frame}
We evaluate the Nash equilibria of these games with respect to two objectives—
social value and profit. The social value of a state S of the network, Val(S), is
defined to be the total utility of all the agents in the system, specifically, the
total value obtained by all the users, minus the prices paid by the users, plus
the profits (prices) earned by all the ISPs. Since prices are endogenous to the
game, this is equivalent to the total value obtained by all the users, and we will
use this latter expression to evaluate it throughout the paper. The worst such
value over all Nash equilibria is captured by the price of anarchy: the price of
anarchy of the NPG with respect to social value, POA Val , is defined to be the
minimum over all Nash equilibria S in N of the ratio of the social value of the
equilibrium to the optimal achievable value Val  :
min S in N (G,D) Val(S)
POA Val (G, D) =
Val 
Here, Val  is the maximum total value achievable while satisfying all the capacity
constraints in the network (this can be computed by a simple flow LP). Likewise,
POA Pro denotes the price of anarchy with respect to profit:
min S in N (G,D) Pro(S)
POA Pro (G, D) =
Pro 
Here Pro(S) is the total utility of all the ISPs, or the total payment made by
all users. The optimal profit Pro  is defined to be the maximum profit over all
states in which users are at equilibrium, and capacity constraints are satisfied.
In instances with a large price of anarchy, we also study the performance of
the best Nash equilibria and provide lower bounds for it. The price of stability
of a game is defined to be the maximum over all Nash equilbria in the game
of the ratio of the value of the equilibrium to the optimal achievable value. We
use POS Val and POS Pro to denote the price of stability with respect to social
value and profit respectively.
\end{frame}




\subsection{Classic Bertrand Competition}

\subsection{B.C. in NPG}

\begin{frame}
We extend the classic Bertrand model of competition to network pricing. The
NPG has two stages. In the first stage, each ISP (edge) e picks a price n e . In the
second stage each user picks paths between its source and destination to send
its traffic. We assume that users can split their traffic into infinitesimally small
chunks, and spread it across multiple paths, or send fractional
P amounts of traffic.
Each user picks paths to maximize her utility, u = v - min P e in P e , where the
minimum is over all paths P from the user’s source to its destination, and v is
its value (or sends no flow if the minimum total price is larger than its value v).
This selection of paths determines the amount of traffic f e on each edge. ISP e’s
utility is given by f e e if f e  c e , and  otherwise. ISPs are selfish and set
prices to maximize their utility.
\end{frame}


\section{Nash Equilibrium}

\subsection{NPG (1 source / 1 sink)}

\begin{frame}
Definition 1. An edge in a given network is called a monopoly if its removal
causes the source of a commodity to be disconnected from its sink.
No monopoly. In the absence of monopolies, the behavior of the network is
analogous to competition in single-item markets. Specifically, competition drives
down prices and enables higher usage of the network, thereby obtaining good
social value but poor profit.
Theorem 1. In a single commodity network with no monopolies, POA Val = 1.
Furthermore, there exist instances with POS Pro = O(L).
\end{frame}



\begin{frame}
Single monopoly. As we show below, the best-case and worst-case performance
of single monopoly networks is identical to that of single-link networks.Theorem 2. In a single commodity network with 1 monopoly, POA Pro = 1 and
POA Val = O(log L). Moreover, there exist instances with POS Val = O(log L).
\end{frame}

\begin{frame}
Multiple monopolies. The performance of the game with multiple monopolies
degrades significantly – the price of anarchy can be unbounded even with 2
monopolies. As we show below, the best Nash equilibrium behaves slightly better
but is still a polynomial factor worse than an optimal solution.
Theorem 3. For every B, there exists a single-source single-sink instance of
the NPG containing 2 monopolies, with L = 2, and POA Val , POA Pro = omega(B).
\end{frame}


\subsection{NPG ( n source / 1 sink)}


\section{Conclusion}

\begin{frame}{Results}
\end{frame}

\begin{frame}{Strengths and weaknesses}
\end{frame}




\end{document}
